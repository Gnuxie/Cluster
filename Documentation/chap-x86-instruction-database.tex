\chapter{X86 instruction database}

\sysname{} contains a database with the description of a significant
number of instructions.  Each instruction is described by one or more
occurrences of calls to the macro \texttt{define-instruction}
described below.  A set of calls is recognized as containing different
variations on the same instruction by the fact that the required
parameter \texttt{mnemonic} is the same string for each member of the
set.

\defmacro define-instruction {mnemonic \key 
				       modes
				       operands
				       opcodes
				       opcode-extension
				       encoding
				       lock
				       operand-size-override
				       rex.w}

\section{Interpreting Intel instruction reference pages}

The instruction reference pages provided by Intel contain complete
descriptions of how each instruction and its variants are encoded.
This key to understanding this description is given in section 3.1 in
the Intel manuals.  Here, we give a more direct description of the
correspondence between the Intel reference pages and the arguments
that should be supplied to the \texttt{define-instruction} macro.

\subsection{Opcodes}

The \texttt{define-instruction} macro has a keyword argument
\texttt{:opcodes}.  This argument should be a list of unsigned octets.
It is preferable to use hexadecimal notation for the opcodes.  The
Intel instruction reference pages contain a column labeled either
``Opcode'' or ``Opcode/Instruction''.  The opcodes are indicated in
that column as a sequence of hexadecimal values, sometimes followed by
``/n'' (where n is a small non-negative integer) or ``/r''.

Sometimes, the list of opcodes is preceded by ``REX.W+''.  When that
is the case, the \texttt{:rex.w} keyword argument should be provided,
with a value of \texttt{t}.

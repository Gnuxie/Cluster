\chapter{X86 instruction database}

\sysname{} contains a database with the description of a significant
number of instructions.  Each instruction is described by one or more
occurrences of calls to the macro \texttt{define-instruction}
described below.  A set of calls is recognized as containing different
variations on the same instruction by the fact that the required
parameter \texttt{mnemonic} is the same string for each member of the
set.

\defmacro define-instruction {mnemonic \key 
				       modes
				       operands
				       opcodes
				       opcode-extension
				       encoding
				       lock
				       operand-size-override
				       rex.w}

\section{Interpreting Intel instruction reference pages}

The instruction reference pages provided by Intel contain complete
descriptions of how each instruction and its variants are encoded.
This key to understanding this description is given in section 3.1 in
the Intel manuals.  Here, we give a more direct description of the
correspondence between the Intel reference pages and the arguments
that should be supplied to the \texttt{define-instruction} macro.

\subsection{Modes}

The Intel documentation contains two columns mentioning \emph{modes}.
One column says ``64-bit Mode'' and the other says ``Compat/Leg Mode''
(which means compatibility or legacy mode).  This information is
reflected in the keyword argument \texttt{:modes} to
\texttt{define-instruction}.  The value of this keyword argument is a
list of one or two elements each of which is either the integer 32 or
the integer 64.  When the Intel documentation mentions ``Valid'' in
the column ``64-bit Mode'' the list should contain 64.  When the
column ``Compat/Leg Mode'' mentions ``Valid'', then the list should
contain 32.

\subsection{Operands}

The \texttt{define-instruction} macro has a keyword argument
\texttt{:operands}.  The argument should be a list of possible
operands.  Valid operands are:

\begin{itemize}
\item (gpr-a 8).  The lower 8 bits of register A, also referred to as AL.
\item (gpr-a 16).  The 16-bit register AX.
\item (gpr-a 32).  The 32-bit register EAX.
\item (gpr-a 64).  The 64-bit register RAX.
\item (gpr 8).  Any 8-bit register.
\item (gpr 16).  Any 16-bit register.
\item (gpr 32).  Any 32-bit register.
\item (gpr 64).  Any 64-bit register.
\item (imm 8).  An 8-bit immediate operand.
\item (simm 8).  An 8-bit immediate operand, sign extended to the size
  of the other operand.  This operand can occur only as the second of
  two operands where the first operand has a size that is greater than
  8 bits.
\item (imm 16).  A 16-bit immediate operand.
\item (simm 16).  A 16-bit immediate operand, sign extended to the size
  of the other operand.  This operand can occur only as the second of
  two operands where the first operand has a size that is greater than
  16 bits.
\item (imm 32).  A 32-bit immediate operand.
\item (simm 32).  A 32-bit immediate operand, sign extended to the size
  of the other operand.  This operand can occur only as the second of
  two operands where the first operand has a size that is greater than
  32 bits.
\item (imm 64).  A 64-bit immediate operand.
\item (memory 8).  An 8-bit memory operand.
\item (memory 16).  A 16-bit memory operand.
\item (memory 32).  A 32-bit memory operand.
\item (memory 64).  A 64-bit memory operand.
\end{itemize}

\subsection{Opcodes}

The \texttt{define-instruction} macro has a keyword argument
\texttt{:opcodes}.  This argument should be a list of unsigned octets.
It is preferable to use hexadecimal notation for the opcodes.  The
Intel instruction reference pages contain a column labeled either
``Opcode'' or ``Opcode/Instruction''.  The opcodes are indicated in
that column as a sequence of hexadecimal values, sometimes followed by
``/n'' (where n is a small non-negative integer) or ``/r''.  The
occurrence of ``/r'' does not need to be encoded in the
\texttt{define-instruction} form.

Sometimes, the list of opcodes is preceded by ``REX.W+''.  When that
is the case, the \texttt{:rex.w} keyword argument should be provided,
with a value of \texttt{t}.

\subsection{Encoding}

The \texttt{:encoding} keyword argument to \texttt{define-instruction}
is a list of the same length as the one supplied to
\texttt{:operands}.  Each element of the list is an encoding.  For
each operand, the encoding indicates how that operand is encoded in
the binary version of the instruction.  Possible values for an
encoding are:

\begin{itemize}
\item \texttt{modrm}.  This value indicates that the operand is encoded in the
  ``mod'' and ``r/m'' fields of the instruction.  It should be used
  when the corresponding entry in the Intel documentation indicates
  r/m8, r/m16, r/m32, or r/m64.
\item \texttt{reg}.  This value indicates that the operand is encoded
  in the ``reg'' field of the instruction.  It should be used when the
  corresponding entry in the Intel documentation indicates r8, r16,
  r32, or r64.
\item \texttt{imm}.  This value indicates that the operand is encoded
  as an immediate value in the instruction stream.  It should be used
  when the corresponding operand in the \texttt{:operands} keyword
  argument mentions \texttt{(imm 8)}, \texttt{(imm 16)}, \texttt{(imm
    32)}, or \texttt{(imm 64)}.
\item \texttt{label}.  This value indicates that the corresponding
  operand is the target of a control-transfer instruction.  It should
  be used when the corresponding operand in the \texttt{:operands}
  keyword argument mentions \texttt{(label 8)}, \texttt{(label 16)},
  \texttt{(label 32)}, or \texttt{(label 64)}.
\item \texttt{-}.  This value indicates that the corresponding operand
  is implicitly defined by the opcode, which is often the case when
  general-purpose register A is one of the operands of the
  instruction.
\item \texttt{+r}.  This value use used in some cases when the
  corresponding operand is a register.  It means that some small
  integer that indicates a register is added to the opcode itself.
  This value should be used when the Intel documentation has a +rb,
  +rw, or +rd in the opcode column.
\end{itemize}

\subsection{Operand-size override}

The keyword argument \texttt{:operand-size-override} to
\texttt{define-instruction} is required to have a true value when the
particular operand requires an instruction prefix in order for that
size operand to be used.  In 32-bit and 63-bit mode, the default
operand size is 32-bits, so for instance if the \texttt{push}
instruction is to be given a 16-bit value instead of a 32-bit value in
one of these modes, then the operand-size-override prefix must be
given.  For this prefix to be emitted, this keyword must be supplied
with a true value.

\subsection{Atomic execution}

When the instruction supports atomic execution, the keyword
\texttt{:lock} with a true argument should be given to
\texttt{define-instruction}.  This value then indicates that the
instruction is capable of executing atomically.

\subsection{64-bit prefix}

In order to access more than 8 registers, Intel instructions require a
prefix called REX.W.  When the Intel documentation mentions this
prefix in the opcode column, the keyword argument \texttt{rex.w}
whould be supplied to \texttt{define-instruction} with a true value.
